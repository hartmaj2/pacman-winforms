\documentclass[a4]{article}
\usepackage[utf8]{inputenc}
\usepackage[czech]{babel}
\usepackage[a4paper,top=2cm,bottom=2cm,left=3cm,right=3cm,marginparwidth=1.75cm]{geometry}
\usepackage{graphicx}
\usepackage{fancyhdr}
\usepackage{hyperref}

\hypersetup{colorlinks=true,linkcolor=black,pdftitle={Vážení kaštanů - Doprovodný dokument},}

\graphicspath{{images/}}

\title{Czech-Man
\\Doprovodný dokument}
\author{Autor: Jan Hartman
\\Garant: RNDr. Tomáš Holan, Ph.D.
\\\\Studium: bakalářské, 1. ročník, letní semestr 2022/2023
\\Předmět: Programování 2}

\date{\today}

\begin{document}
\maketitle
\tableofcontents
\pagestyle{fancy}
\newpage
\section{Úvodní část}
\subsection{Anotace}
Program Czech-Man je napodobeninou klasické hry Pac-Man z roku 1980. Neklade si za cíl být naprosto přesnou kopií hry, hlavní funkčnosti by měly být zachovány. Hra je vyvíjena v programovacím jazyce C\#.
\subsection{Formát tohoto dokumentu}
Pokud se v tomto dokumentu budu vyjadřovat k nějaké třídě nebo objektu jejím jménem, jako je v programu tak budu psát \textbf{tučně}. Jako třeba pokud zmiňuji třídu \textbf{GameManager}.
\\\\
Naopak pokud zmiňuji nějaké klíčové slovo jazyka C\# nebo nějakou built-in třídu, tak budu používat formát \verb|verbatim|. Například \verb|char| nebo \verb|Bitmap|.
\subsection{Zvolené vývojové prostředí}
Prvotním úmyslem bylo hru vyvíjet ve vývojovém prostředí Unity, které je přímo určeno k vytváření počítačových her. S Unity jsem však neměl moc velké předchozí zkušenosti a po krátké snaze naučit se základy fungování tohoto prostředí jsem jeho užití zavrhl, jelikož naučit se v tomto prostředí pracovat vyžaduje mnoho specifických znalostí. Zároveň mi prostředí Unity nevyhovovalo v jeho zaměření na přesné fyzikální simulace a také fakt, že se nedá toto zaměření plně „odstínit“. Chtěl jsem pracovat v takovém prostředí, kde budu mít nad svým herním světem co největší kontrolu, a tak jsem se nakonec rozhodl pro složitější postup, který mi však umožnil mít více kontroly nad programem.
\\\\
Touto volbou bylo pracovat v prostředí Windows Forms (dále už jen WinForms). To má zase nevýhodu v tom, že není primárně určeno pro tvorbu počítačových her, ale spíš k vytváření převážně statických okenních aplikací. Tím jsem se však nenechal odradit, jelikož žádné jiné prostředí podporující grafické prvky jsem neuměl používat a trvalo by mi dost času se s jakýmkoliv jiným naučit. Byla to tedy čistě pragmatická volba.
\newpage
\section{Programátorská část}
\subsection{Hlavní principy sledované při tvorbě programu}
\subsubsection{Obecnost a přenositelnost kódu do budoucna}
Při psaní kódu jsem se snažil hlavně dbát na to, aby byl program psán co nejvíce obecně a tudíž abych měl co největší prostor program do budoucna rozvíjet jakýmikoliv směry. Chtěl jsem si také nechat otevřenou možnost v budoucnu přejít na jiné vývojové prostředí než WinForms a dbal jsem tedy na to, ať je na WinForms všechna logika hry kromě samotného vykreslování co nejvíce nezávislá. Toto mě vedlo například k úplnému odstranění vizuálního designeru formy, jelikož jsem chtěl co nejvíce funkčnosti obsažené v samotném kódu, ze kterého by se v budoucnu dalo vyčíst, jak program převést do jiného prostředí lépe než z interních nastavení WinForms.
\\\\
Také mě to vedlo k implementaci herní smyčky jiným způsobem než užitím třídy \verb|Timer| poskytovanou prostředí WinForms (údajně není pro herní smyčky tak přesný). Zároveň je to další prvek spojený přímo s WinForms a bylo tedy lepší ho nahradit užitím C\# třídy \verb|Stopwatch| a \verb|TimeSpan| k spočtení, jestli se má hra občerstvit. Tím jsem se zbavil další závislosti na WinForms.
\subsubsection{Objektová orientace}
Snažil jsem se také programovat co možná nejvíce objektově. Dával jsem tedy důraz na to, aby jakákoliv věc figurující ve hře byla reprezentována nějakým objektem. Zde jsem možná zaběhl do přílišných extrémů a můj kód je proto delší než by mohl být. Například tím, že jako objekty reprezentuji i prázdná políčka (což má i nějaké výhody). Upřednostňuji však často i delší kód, pokud je to za cenu lepší přehlednosti kódu. Tím, že téměř každou věc reprezentuji objektem si také otevírám dveře k tomu, abych v budoucnu mohl těmto věcem přiřadit nějaké chování, což určitě není k zahození.
\subsubsection{Čitelný kód}
Dalším principem, kterého jsem se snažil držet byl důraz na tzv. „samokomentující“ kód. To znamená, že jsem dbal velmi na to, aby jména proměnných byla co nejvíce výstižná a opravdu popisovala, co dané objekty, proměnné nebo funkce znamenají či dělají. Nebál jsem se proto klidně i delších jmen pokud mi přišlo, že by to pomohlo budoucímu čtenáři se v kódu lépe zorientovat. Zde jsem se však přistihl, že možná také zabíhám do přílišných extrémů. Během tvorby programu jsem si také uvědomil, že nemá smysl ve jménu proměnné slepě opakovat typ proměnné, ale snažit se trefným pojmenováním objektu v aktuálním kontextu přidat informační hodnotu navíc. Příkladem může být třeba příkaz: 
\\\\
\verb|List<StaticLayerBlankSpace> adjacentExits = new List<StaticLayerBlankSpace>();|
\\\\
Zde by nemělo smysl pojmenovávat seznam jako \textbf{adjacentBlankSpaces} jelikož fakt, že jde o prázdné políčko už vyplývá z typu proměnné.
\\\\
S čitelností kódu souvisí také využití privátních metod, které jsou volány jen jednou a mohou se tedy zdát zbytečné. Mají však dle mého názoru roli podobnou, jako by měl komentář popisující celou takovou část kódu. Krásně se tím vystihne funkce daného úseku jako logického celku a umožní do budoucna třeba využít tento logický celek někde jinde v kódu. Dobrým příkladem budiž třeba metoda \textbf{SetAllGhostsToFrightenedIfPossible()} třídy \textbf{GameManager}. (Ano, je to trochu extrém.)
\\\\
Snažil jsem se také chytře využít klíčová slova jako \verb|private|, \verb|public| nebo \verb|protected| k tomu, aby bylo zřejmé, jaký má daná funkce či atribut charakter tzn. jestli má být využit pouze interně a jde o jakýsi pomocný prvek nebo je to klíčový element, který slouží ke komunikaci mezi jednotlivými objekty programu. Podobným způsobem jsem oddělil data, která se mění od těch, která po dobu života objektu zůstávají neměnná klíčovým slovem \verb|readonly|. Tyto data jsem také označil jako veřejná, jelikož mi nedávalo smysl pro ně zavádět „gettery“, když stejně vrací stále to samé a nemusí se nic počítat. Dobrým příkladem jsou například atributy objektu \textbf{Map}, které souvisí s neměnnou velikostí mapy (třeba \textbf{gridWidth} nebo \textbf{cellSize}).
\subsubsection{Zapouzdření privátních dat}
V neposlední řadě jsem pak usiloval o to, aby každý objekt zpřístupňoval navenek jen ty informace, které jsou potřeba a více ne. To se vyplácí nejen z důvodu lepší čitelnosti, jak zmiňuji výše, ale také to usnadňuje práci s našeptávačem, který mi zbytečně nenabízí ty funkce, které stejně nebudu chtít na daném objektu nikdy volat.
\subsection{Chronologický postup tvorby}
Zde se pokusím popsat, jak se program postupně vyvíjel a jak na sobě jednotlivé důležité části programu časově navazují. Mám pocit, že je to užitečná informace, která slouží k lepšímu pochopení, proč jsou různé funkcionality programu řešeny tak, jak jsou řešeny a poskytne lepší přehled o tom, jaké části programu jsou závislé na kterých ostatních.
\subsubsection{Načítání vstupních dat}
Nejprve bylo nutné vůbec se rozhodnout, jak budu reprezentovat vstupní data. Rozhodl jsem se, že nejjednodušší bude zadávat herní mapu jako textový soubor, ve kterém jednotlivé symboly budou znamenat různé herní objekty. Rozhodl jsem se nejprve zadávat soubory jako Resources. (Což umožňuje Visual Studio a poskytuje možnost dané nahrané zdroje, jako třeba obrázky přímo adresovat v kódu svým jménem jako by to byla klasická proměnná) To jsem však později změnil, jelikož mi přišlo příliš krkolomné přidávat takto nové obrázky. (Bylo potřeba otevřít soubor s příponou .resx a proklikat se několika volbami.) Je jednodušší mít jednu složku, ze které akorát přečtu soubor s daným jménem, které zadám do proměnné jako \verb|string|.
\\\\
Jinak jsem se snažil vstupní data co nejvíce oddělit od zbytku programu a založil jsem tedy samostatnou statickou třídu \textbf{GamePresets}, která se stará o vše týkající se načítání a předzpracování vstupních dat. Tím je myšleno například vytvoření hotového objektu \textbf{Map} z přečteného souboru \textbf{map.txt}, který obsahuje textové zadání herního plánku. Nebo třeba vytvoření objektů typu \verb|Bitmap|, které pak předávám jednotlivým instancím třídy \textbf{GameObject} jako jejich obrazové reprezentace (neboli „sprity“).
\\\\
V prvotních fázích programu si objekty tyto svoje interní informace načítaly přímo od třídy \textbf{GamePresets}, ale toto „šahání jiné třídě do zelí“ se mi později znelíbilo a zvolil jsem raději předávání těchto informací jako parametry konstruktoru. To byl také případ objektu \textbf{Map}, jehož seznam parametrů však na druhou stranu touto úpravou nabobtnal na dobrých devět prvků!
\\\\
To, že si později každý objekt pamatoval všechna data jenž se ho týkala, vedlo i k tomu, že jsem nemusel ve třídě \textbf{Painter} provádět složité kontrolování toho, jaký objekt se to vlastně teď snažím nakreslit pomocí \verb|if|ů, jak popisuji v následující podsekci.
\subsubsection{Základní grafika}
Dále bylo nutné se postarat o to, abych dokázal vykreslit herní pole se všemi objekty které v něm žijí. Bez toho by vůbec nemělo smysl implementovat nějaké chování objektů. Zde jsem se inspiroval kódem, který jsme využívali na cvičeních z Programování 2. (konkrétně při modifikaci hry s padajícími balvany a diamanty) Rozhodl jsem se však pro větší obecnost a tudíž místo aby si informace o tom, jaké obrázky je potřeba vykreslit držela třída \textbf{Painter}, tak má každý herní objekt metodu \textbf{GetImageToDraw()}, která vrátí jeho aktuální bitmapu (tedy objekt třídy \verb|Bitmap|) k vykreslení. To mi umožní, aby objekt sám mohl podle svého stavu vykreslit bitmapu kterou potřebuje a \textbf{Painter} se o to tedy vůbec nemusí starat a nemusím mít žádný \verb|switch|, který by rozhodoval, jaký objekt to je a co mám tedy vlastně vykreslit. To se mi konkrétně hodilo při vykreslování duchů v závislosti na jejich aktuálním módu. Pokud duchové byli vyděšení a měli prchat, tak stačilo přepsat rodičovskou metodu \textbf{GetImageToDraw()} a rozhodnout se, jakou bitmapu vrátit na základě aktuálního nastaveného módu (\textbf{currentMode}).
\subsubsection{Vykreslování pohybujících se objektů}
Poté co jsem zprovoznil vykreslování nehybných objektů přišlo na řadu vyzkoušet, jak si program povede s vykreslováním pohybujících se objektů. Zde jsem v první řadě zapomněl vždy vyčistit komponentu \verb|Graphics| a tudíž se mi kreslily další vrstvy přes sebe. Nabízelo se snadné řešení: Před každým dalším vykreslením všech objektů smazat celé plátno. S tím se však objevil nový problém, jelikož celá obrazovka při každém přepsání blikala. Zde mi hodně pomohla ChatGPT, která mi poradila, že to nejspíš bude tím, že program nestíhá tak rychle občerstvovat plátno a že bych měl kreslit jednotlivé obrázky do bufferu (reprezentován objekty \textbf{bufferBitmap} a \textbf{bufferGraphics}) teprve až budou všechny připravené, tak výsledek překreslit na plátno. (K tomu slouží příkaz \verb|formGraphics.DrawImageUnscaled(bufferBitmap,0,0)|, kde \textbf{formGraphics} je \verb|Graphics| komponenta našeho herního okénka vytvořená příkazem \verb|CreateGraphics()|.) O vyčištění bufferu před dalším malováním se stará funkce \textbf{ClearBuffer()} a o překopírování bufferu do herního okénka pak \textbf{WriteBuffer()}.
\subsubsection{Rozdělení projektu do více souborů}
V tomto momentu, kdy jsem se pro cokoliv, co tomu alespoň trochu nahrávalo rozhodl vytvořit objekt už jsem měl opravdu mnoho kódu. Uvědomil jsem si, že mi poměrně dlouho zabere samotné ježdění nahoru a dolu po mém zdrojovém souboru a hledání té správné třídy, jejíž kód jsem zrovna potřeboval upravit. To byla veliká motivace k tomu, založit si pro můj projekt více souborů, které by logicky odpovídaly různým částem programu. Tak jsem se tedy rozhodl pro to, mít pět hlavních zdrojových souborů. Těmi byly následující:
\begin{itemize}
    \item \textbf{GameForm.cs}
    \item \textbf{GameManager.cs}
    \item \textbf{GameObjects.cs}
    \item \textbf{HelperClasses.cs}
    \item \textbf{InputManager.cs}
\end{itemize}
Soubor \textbf{GameForm.cs} obsahuje kód nutný k inicializaci okénka hry a předání informací o okénku třídě \textbf{GameManager}, která žije ve vlastním stejnojmenném souboru. Tato třída se pak stará o samotnou logiku hry. Rozhoduje o tom, co se má vykreslovat na základě toho, v jakém je hra stavu, implementuje logiku herní smyčky a přijímá od Windows Forms stisknuté klávesy od uživatele, podle kterých se pak rozhoduje, co dělat. V neposlední řadě si třída \textbf{GameManager} drží reference na dva pro běh hry klíčové objekty \textbf{Map} a \textbf{Painter}.
\\\\
Tyto dva objekty žijí ve vlastním souboru jménem \textbf{HelperClasses.cs} společně se strukturou \textbf{Direction}. To jsou v podstatě třídy, které mi neseděly do jiného souboru. V budoucnu možná ještě tyto třídy nějak logicky rozdělím. 
\\\\
Dalšími souborem je \textbf{GameObjects.cs}, který obsahuje kód všech objektů, které žijí ve hře, dají se nějak vykreslit (většinou), dá se s nimi nějak interagovat, nebo do nich narážet a nebo se samy pohybují. Tyto objekty všechny dědí od společné rodičovské třídy \textbf{Game Object}. Ta poskytuje abstrakci nad všemi objekty a vynucuje si například, aby obsahovaly implementaci metod \textbf{GetGridX()} a \textbf{GetGridY()}, které slouží k vrácení aktuální souřadnic v mřížce (a to i pokud se objekt zrovna pohybuje mezi dvěmi buňkami mřížky!). Také si vynucuje, aby každý objekt o sobě uměl říci, jestli je vykreslitelný (metoda \textbf{IsDrawable()}) nebo aby uměl vrátit svou bitmapu, jak jsem již zmiňoval.
\\\\
Posledním souborem, který ještě zbývá zmínit je soubor \textbf{GamePresets}, který se stará o předvolby programu jako jsou třeba rychlost hráče, velikost jedné buňky herního pole, nebo samotné čtení zadání herní mapy ze souboru \textbf{map.txt} či čtení obrázků ze souborů ve formátu PNG. Stará se tedy o předzpracování tohoto vstupu, aby si ho všechny objekty mohly jednoduše přečíst a nemusely se starat o nic dalšího. Tímto rozdělením se dá snadno všechny předvolby programu nastavovat pěkně na jednom místě a není třeba prohledávat kódy jednotlivých tříd.
\\\\
Druhým důvodem pro vytvoření vlastní třídy pro veškeré obstarávání vstupu (téměř veškeré až na přednastavení textu a ostatních věcí souvisejících více s WindowsForms než s mechanikou hry), kdybych se někdy rozhodl vstupní data načítat jiným způsobem. Pak bude stačit jednoduše upravit pouze tento soubor a nikde jinde nebude potřeba dělat skoro žádné změny.
\subsubsection{Pohyb objektů mimo mřížku}
Velikým zádrhelem, který jsem zprvu naivně úplně opomenul byl fakt, že i přesto, že hra Pac-Man se vlastně celá odehrává na 2D mřížce, tak se jednotlivé objekty, jako samotný Pac-Man nebo duchové pohybují i mimo mřížku a jednotlivé buňky mřížky jim slouží pouze jako jakési „křižovatky“, kde občas mohou zatočit. Zde jsem nevymyslel žádné chytřejší řešení, než si spočítat velikost buňky v pixelech a dát si vždy pozor na to, aby rychlost objektu dělila tuto velikost beze zbytku. (Třída, od které objekty mající tuto vlastnost dědí se jmenuje \textbf{TweeningObject}.) Tím jsem zaručil, že po určitém počtu snímků se mi postavička vždy objeví přesně na další buňce a můžu tedy v kódu ošetřit, co dělat dál. Jestli například sníst nějakou kuličku (\textbf{Pellet}), nebo třeba zatočit někam jinam, či si vůbec zkontrolovat, jestli mohu pokračovat dál (\textbf{CanGoInDirection()}).
\\\\
Zde jsem využil také skryté funkčnosti, která může být pro někoho trochu překvapivá a tím je, že pokud nastavíme postavičce rychlost na nějakou, která by nevyhovovala požadavku dělitelnosti zmíněného výše, tak bude automaticky snížena na první takovou, která už bude velikost buňky dělit beze zbytku. To může být trochu překvapující (až matoucí), ale rozhodl jsem se tak pro to, abych mohl rychle testovat různé rychlosti a nemusel jsem vždy v hlavě počítat, jestli to zrovna bude fungovat nebo ne. Konkrétně se o tuto funkčnost stará metoda \textbf{SetTweenSpeed()}. 
\\\\
Na druhou je tato redukce rychlosti provedená další pomocnou funkcí \textbf{ReduceToDivisibleWithoutRemainder()} a v kódu volající funkce je tedy dobře tato volaná funkce vidět. To mi umožňuje, abych ji pokud to bude nutné mohl v budoucnu jednoduše najít a odstranit.
\subsubsection{Přednastavování hráčova směru}
Dále jsem chtěl naimplementovat herní mechaniku, která hráči umožní přednastavit, kterým směrem se hráčova postavička vydá příště, i když to zrovna není možné. Pokud tedy například stisknu šipku nahoru, tak si postavička tento směr zapamatuje a pokusí se v nejbližší možné době tímto směrem vydat. To však nebylo nijak složité. Stačilo přidat atribut \textbf{nextDirection}, který umožňuje si daný směr do příště zapamatovat a využít to pokud možno při testování, kudy by se měla postavička z aktuální buňky pohnout dále. (To se děje uvnitř metody \textbf{StartNextMovementCycle()}.)
\subsubsection{Detekce kolizí s duchy}
Implementace detekce kolizí s duchy mě trochu děsila. Nevěděl jsem, zdali mi v tom skutečnost, že se jak hráč, tak i duchové pohybují mimo mřížku nebude dělat paseku. Nakonec jsem si však uvědomil, že stačilo jednoduše zjistit, zdali se jak x-ová, tak y-ová souřadnice (v pixelech!) obě liší maximálně o velikost jedné buňky. Pokud ano, tak se objekty dotýkají, jinak ne. Tento výpočet provádí funkce \textbf{IsTouchingTweeningObject()}. V hlavní smyčce hry tedy jednoduše zavolám funkci \textbf{IsTouchingAnyGhost}(), která na vstupu přijme seznam všech duchů ve hře a pomocí výše zmíněné funkce otestuje, jestli se některého z nich hráč dotýká nebo ne.
\subsubsection{Implementace tzv. „wraparound“ a problémy s tím spojené}
Pokud se postavička pokusí opustit herní pole, tak bylo potřeba, aby se objevila přesně na druhé straně mapy. To nebylo pro pohyb hráče tak složité. Bylo však nutné hráčovy souřadnice slepě neměnit po přetečení nebo podtečení přesně na nulu nebo maximální hodnotu, ale vždy tuto hodnotu snížit nebo zvýšit o nějaký násobek velikosti jedné buňky. Jinak by se totiž mohlo stát, že by postavička ukončila jeden svůj cyklus pohybu mimo střed nějaké buňky. To by mohlo mít za následek, že by pak postavička skončila pohyb uvnitř zdi nebo jinou podobnou prekérní situaci. O toto „zaobalování“ souřadnic se stará funkce \textbf{GetWrappedPixelLocation()}, kterou poskytuje třída \textbf{Map}.
\\\\
Toto fungovalo nějakou dobu uspokojivě, ale později se ukázalo, že duchové, kteří pro své fungování potřebují znát mřížkovou lokaci poslední navštívené buňky při přechodu na druhý okraj mapy neregistrovaly krajní políčka mapy. To vedlo k tomu, že si mysleli, že toto políčko ještě nenavštívili a vraceli se zpět. Toto jsem bohužel musel opravit jako speciální případ (nenašel jsem lepší způsob) a řeším ho v metodě \textbf{WraparoundIfOutOfBounds()} zavoláním metody \textbf{UpdateLastOccupied()}, kterou bych ve standardním případě volal jen před začátkem dalšího cyklu pohybu mezi dvěmi sousedními buňkami.
\\\\
Aby toho nebylo málo, tak se objevil další nečekaný zádrhel. Duchové totiž při hledání cesty, kterou se mají bludištěm navigovat kontrolují sousední políčka a koukají se, která z nich jsou prázdná. Co když jsou však duchové na kraji mapy? Nabízí se jednoduše „zaobalit“ souřadnici souseda při kontrole všech sousedních políček abychom získali políčko, které je sice na druhé straně mapy, ale prakticky je sousední. Zde jsem však zkolaboval na předpokladu, že poté co si duch zvolí sousední políčko na které bude pokračovat, stačí odečíst jeho aktuální pozici od pozici souseda a získám tím požadovaný vektor příštího pohybu. Výsledkem byl duch, jehož vektor \textbf{direction} (To je proměnná typu \textbf{Direction}, což je mnou vytvořený \verb|struct|, který reprezentuje směr pohybujících se postaviček) měl místo klasické jednotkové délky délku rovnou velikosti mapy (tedy buď šířky nebo délky). Hodnota privátní proměnné \textbf{direction} se však využívá uvnitř těla metody \textbf{ContinueMoving()} objektu \textbf{TweeningObject} k výpočtu toho, o kolik pixelů se v daném snímku hry má duch posunout. Tak se stalo, že mi duchové lítali z jedné strany mapy na druhou bleskovou rychlostí dokud je nezastavila zákeřná \verb|IndexOutOfBoundsException|. Řešením byl bohužel již druhý speciální případ, který jsem musel umístit dovnitř metody \textbf{SetDirectionTowardsExit()}, kterou poskytuje třída \textbf{MovingObject}. Konkrétně dle toho, kterým směrem se vektor protáhl, lze určit správný směr, kterým by se měl duch vydat. Toť směr k tomuto přerostlému směru přesně opačný avšak jednotkové délky. Toto jednoduché řešení poskytuje privátní metoda \textbf{GetFixedOutOfBoundsCoordinate()}, která upraví zvlášť každou složku takto zdivočelého vektoru.
\subsubsection{Přidání hacku na rychlejší a ortodoxnější herní smyčku ve WinForms}
Dále jsem měl pocit, že hra běží na mém počítači poněkud pomalu a občas se zadrhává, což bylo z velké části nějspíš i tím, že jsem k práci s WinForms musel použít virtuální stroj k emulaci operačního systému Windows. Rozhodl jsem se tedy probádat internet, zdali někdo nemá řešení, jak program ve WinForms zrychlit. 
\\\\
Našel jsem způsob, který využívá jakýsi hack, kdy program nečeká, dokud se nespustí nějaký jeho event, ale namísto toho běží neustále, dokud je tzv. „windows message queue“ prázdná. Tento způsob je kromě (údajné) vyšší rychlosti a přesnosti herní smyčky nezávislý na tikání WinFormsového timeru (objekt typu \verb|Timer|), který se údajně občas trochu zpožďuje a není tak přesný. 
\\\\
Musím se však přiznat, že jsem nezaznamenal žádné očividné zlepšení avšak ani zhoršení. Rozhodl jsem se tedy tuto změnu v programu ponechat. Částečně také z toho důvodu, že nová implementace se více podobá klasické herní smyčce, která se běžně užívá v počítačových hrách a navíc je závislá na objektu typu \verb|Stopwatch|, který by měl poskytovat to zmíněné pravidelnější tikání.
\subsubsection{Počátky umělé inteligence duchů}
Dále jsem se rozhodl konečně uskutečnit zárodky umělé inteligence duchů. Tomu předcházelo, že jsem si přečetl jeden \href{https://www.gamedeveloper.com/design/the-pac-man-dossier}{pěkný článek} o tom, jak se implementovala tato umělá inteligence v originálním Pac-Manovi. Zjistil jsem, že jádro chování jednotlivých duchů je výpočetně jednodušší, než jsem se původně bál a nebudu tedy muset programovat žádné prohledávání do šířky na každé křižovatce a vystačím si jednoduše s počítáním vzdáleností. 
\\\\
Algoritmus hledání cesty bludištěm totiž spočívá v tom, že každý duch má v každém čase programu nějaký svůj cíl (v programu ho reprezentuji proměnnou \textbf{target} typu \verb|Point|), na který se aktuálně snaží dostat (jeho poloha se neustále mění v čase). Pro volbu, kterou cestou se vydá na příští křižovatce pak duch volí jednoduchý hladový přístup, který spočívá v tom, že se koukne na všechny sousední políčka na křižovatce a vybere si takové sousední políčko, které je od jeho zvoleného cíle nejbližší vzdušnou čarou. A to nehledě na skutečný počet buněk, které bude muset duch při pronásledování svého cíle překonat. Kupodivu tento přístup celkem pěkně funguje (pokud předpokládáme klasickou Pac-Manovskou herní mapu).
\subsubsection{Stavy hry}
Abych mohl pracovat na složitějším chování duchů, tak bylo nejprve potřeba přidat základní herní stavy. Tedy počáteční obrazovku, běžící hru a konec hry. Chtěl jsem totiž mít jednoduchou možnost otestovat chování duchů a nemuset hru stále dokola spouštět manuálně přes Visual Studio. Místo toho jsem chtěl mít možnost jednoduše zmáčknout enter a hrát znovu. Dalším důvodem bylo ladění obtížnosti. To se také dělá těžko, pokud hra ještě nejde vyhrát. Takže bylo nutné tyto stavy přidat.
\\\\
Samotná implementace stavů není složitá. Uvažoval jsem, že bych použil tzv. „strategy pattern“ a využil \verb|interface| a vytvořil tzv. „state machine“. Nakonec mi toto však přišlo moc složité a zvolil jsem jednodušší způsob vytvořením výčtového typu \verb|enum|, který může nabývat hodnot různých stavů hry. Co se má dělat v jakém stavu pak kontroluji pomocí řídící struktury \verb|switch|.
\subsubsection{Módy duchů}
Jakmile jsem měl možnost přecházet mezi stavy hry, tak jsem se pustil do psaní kódu, který duchům umožní přecházet mezi čtyřmi důležitými stavy. Prvním stavem je ten, kdy je duch ještě v domečku. Dalším stavem je tzv. stav „chase“, ve kterém duch (většinou) neúprosně pronásleduje hráče. Třetím stavem je tzv. „scatter“ ve kterém duch prchne na svůj vytyčený roh herního pole. Posledním důležitým stavem je stav „frightened“, ve kterém může hráč ducha sníst za dodatečné body a samotný duch se náhodně pohybuje po herním poli.
\\\\
Módy duchů jsem se rozhodl stejně jako stavy hry řešit pomocí výčtového typu \verb|enum| a \verb|switch|. To, že každý duch má v daném stavu své vlastní unikátní chování pak budu řešit pomocí dědičnosti. Udělám si pro každého ducha vlastní třídu a obecné chování ducha bude volat virtuální metodu, což umožní, aby každý duch tuto virtuální metodu „overridoval“ podle sebe.
\end{document}
