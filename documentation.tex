\documentclass[a4]{article}
\usepackage[utf8]{inputenc}
\usepackage[czech]{babel}
\usepackage[a4paper,top=2cm,bottom=2cm,left=3cm,right=3cm,marginparwidth=1.75cm]{geometry}
\usepackage{graphicx}
\usepackage{fancyhdr}
\usepackage{hyperref}

\hypersetup{colorlinks=true,linkcolor=black,pdftitle={Vážení kaštanů - Doprovodný dokument},}

\graphicspath{{images/}}

\title{Czech-Man
\\Doprovodný dokument}
\author{Autor: Jan Hartman
\\Garant: RNDr. Tomáš Holan, Ph.D.
\\\\Studium: bakalářské, 1. ročník, letní semestr 2022/2023
\\Předmět: Programování 2}

\date{\today}

\begin{document}
\maketitle
\tableofcontents
\pagestyle{fancy}
\newpage
\section{Úvodní část}
\subsection{Anotace}
Program Czech-Man je napodobeninou klasické hry Pac-Man z roku 1980. Neklade si za cíl být naprosto přesnou kopií hry, hlavní funkčnosti by měly být zachovány. Hra je vyvíjena v programovacím jazyce C\#.
\subsection{Formát tohoto dokumentu}
Pokud se v tomto dokumentu budu vyjadřovat k nějaké třídě nebo objektu jejím jménem, jako je v programu tak budu využívat \textit{kurzívu}. Jako třeba pokud zmiňuji třídu \textit{GameManager}.
\\\\
Naopak pokud zmiňuji nějaké klíčové slovo jazyka C\#, tak budu používat formát \verb|verbatim|. Například \verb|char| nebo \verb|class|.
\subsection{Zvolené vývojové prostředí}
Prvotním úmyslem bylo hru vyvíjet ve vývojovém prostředí Unity, které je přímo určeno k vytváření počítačových her. S Unity jsem však neměl žádné předchozí zkušenosti a po krátké snaze naučit se základy fungování tohoto prostředí jsem jeho užití zavrhl, jelikož naučit se v tomto prostředí pracovat vyžaduje mnoho specifických znalostí. Zároveň mi prostředí Unity nevyhovovalo v jeho zaměření na přesné fyzikální simulace a také fakt, že se nedá toto zaměření plně „odstínit“. Chtěl jsem pracovat v takovém prostředí, kde budu mít nad svým herním světem co největší kontrolu, a tak jsem se nakonec rozhodl pro složitější postup, který mi však umožnil mít více kontroly nad programem.
\\\\
Touto volbou bylo pracovat v prostředí Windows Forms (dále už jen Winforms). To má zase nevýhodu v tom, že není zdaleka určeno pro tvorbu počítačových her, ale spíš k vytváření více méně statických okenních aplikací. Tím jsem se však nenechal odradit, jelikož žádné jiné prostředí podporující grafické prvky jsem neuměl používat a trvalo by mi dost času se s jakýmkoliv jiným naučit.
\newpage
\section{Programátorská část}
\subsection{Hlavní principy sledované při tvorbě programu}
Při psaní kódu jsem se snažil hlavně dbát na to, aby byl program psán co nejvíce obecně a tudíž abych měl co největší prostor program do budoucna rozvíjet jakýmikoliv směry. Chtěl jsem také si nechat otevřenou možnost v budoucnu přejít na jiné vývojové prostředí než Winforms a dbal jsem tedy na to, ať je na Winforms všechna logika hry kromě samotného vykreslování co nejvíce nezávislá. Toto mě vedlo například k úplnému odstranění vizuálního designeru formy, jelikož jsem chtěl co nejvíce funkčnosti obsažené v samotném kódu, ze kterého by se v budoucnu dalo vyčíst, jak program převést do jiného prostředí lépe než z interních nastavení Winforms.
\\\\
Snažil jsem se také programovat co možná nejvíce objektově. Dával jsem tedy důraz na to, aby všechno ve hře byl nějaký objekt. Zde jsem možná zaběhl do přílišných extrémů a můj kód je proto delší než by mohl být. Upřednostňuji však vždy delší kód, pokud tím získám lepší přehlednost a znovupoužitelnost kódu. Tím, že téměř každou věc reprezentuji objektem si nezavírám dveře před tím, abych mohl k jakémukoliv takovému objektu později přidat i nějaké chování a to mi přijde jako velmi pozitivní vlastnost.
\\\\
Dalším principem, kterého jsem se snažil držet byl důraz na tzv. „samokomentující“ kód. To znamená, že jsem dbal velmi na to, aby jména proměnných byla co nejvíce výstižná a opravdu popisovala, co dané objekty, proměnné nebo funkce znamenají či dělají. Nebál jsem se proto klidně i delších jmen pokud mi přišlo, že by to pomohlo budoucímu čtenáři se v kódu lépe zorientovat. Zde jsem se však přistihl, že možná také zabíhám do přílišných extrémů. Během tvorby programu jsem si také uvědomil, že nemá smysl ve jménu proměnné slepě opakovat typ proměnné, ale snažit se trefným pojmenováním objektu v aktuálním kontextu přidat informační hodnotu navíc.
\\\\
V neposlední řadě jsem pak usiloval o to, aby každý objekt zpřístupňoval navenek jen ty informace, které jsou potřeba a více ne. To se vyplácí z toho důvodu, že je jasnější, jaký je záměr jaké funkce. Zdali má daná funkce být pouze malou pomocnou funkcí nebo něčím klíčovým, čím objekt komunikuje navenek. Dále to také usnadňuje práci z našeptávačem, který mi zbytečně nenabízí ty funkce, které stejně nebudu chtít na daném objektu nikdy volat.
\subsection{Chronologický postup tvorby}
Zde se pokusím popsat, jak se program postupně vyvíjel a jak na sobě jednotlivé důležité části programu časově navazují. Mám pocit, že je to užitečná informace, která slouží k pochopení, proč jsou různé věci řešeny tak jak jsou řešeny a dá lepší přehled o tom, jaké části programu jsou závislé na jakých jiných.
\subsubsection{Načítání vstupních dat}
Nejprve bylo nutné vůbec se rozhodnout, jak budu reprezentovat vstupní data. Rozhodl jsem se, že nejjednodušší bude zadávat herní mapu jako textový soubor, ve kterém jednotlivé symboly budou znamenat různé herní objekty. Rozhodl jsem se také zadávat soubory pomocí funkce Resources, kterou umožňuje Visual Studio, ale to jsem později změnil, jelikož mi přišlo příliš krkolomné přidávat takto nové obrázky. Je jednodušší mít jednu složku, ze které akorát přečtu soubor s daným jménem. Zároveň aby se projevily změny, tak je při využití Resources nutné také klikat na volbu Clean Build, což také přidává zbytečný krok navíc. 
\\\\
Jinak jsem se snažil vstupní data co nejvíce oddělit od zbytku programu a založil jsem tedy samostatnou statickou třídu \textit{InputManager}, která se stará o vše co se týká načítání a předzpracování vstupních dat.
\\\\
Na načítání vstupních dat bych však také chtěl ještě opravit způsob, kterým jsou vstupní data předávány samotným objektům. Přijde mi lepší, aby objekty dostávaly tyto data jako svoje parametry konstruktoru jelikož tím pádem budou objekty méně závislé na konkrétní implementaci reprezentace vstupních dat.
\subsubsection{Vykreslování herního pole}
Dále bylo nutné se postarat o to, abych dokázal vykreslit herní pole se všemi objekty které v něm žijí. Bez toho by vůbec nemělo smysl implementovat nějaké chování objektů. Zde jsem se inspiroval kódem, který jsme využívali na cvičeních z Programování 2 při upravování hry s balvany. Zde jsem se však rozhodl pro větší obecnost a z důvodu, že nereprezentuji jednotlivá políčka jako proměnné typu  \verb|char| jsem musel použít pro rozhodování, který obrázek vykreslím klíčové slovo \verb|is|. Toto řešení mi však stále přijde trochu neohrabané a chtěl bych v budoucnu řešit toto vypisování tak, že by si každý objekt držel referenci na svou bitmapu a tu pak poskytnul třídě \textit{Painter}, která by pak akorát tuto bitmapu vykreslila.
\\\\
Poté co jsem zprovoznil vykre
\end{document}
